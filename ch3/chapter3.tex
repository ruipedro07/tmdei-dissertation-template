% Chapter 3

\chapter{State of the Art} % Main chapter title
\label{chap:Chapter3} 

%----------------------------------------------------------------------------------------
\section{Methodology}

\subsection{Research Questions}


RQ1- Quais são os desafios técnicos na implementação de uma solução baseada em RAG para suporte técnico?



RQ2 - Quais frameworks ou bibliotecas disponíveis apresentam melhor desempenho, flexibilidade e facilidade de integração em soluções RAG para suporte técnico?




RQ3 - Quais as melhores práticas para a indexação e recuperação de dados? 


RQ4 - Qual a LLM mais adequada para uso de RAG para suporte técnico? 








\subsection{Research Scope}

\subsection{Eligibility Criteria}


\subsection{Selection Process}

\subsection{Data Collection}

\section{Results and Analysis}


\subsection{RQ1- Quais são os desafios técnicos na implementação de uma solução baseada em RAG para suporte técnico?}

Existem diversos desafios técnicos na implementação de uma solução RAG para suporte técnico, tais como para a indexação e recuperação de dados, a integração com sistemas existentes no sentido de alimentação do conhecimento e precisão e relevância das respostas. 

\subsubsection{Indexação, recuperação e relevância das respostas}

A eficiência do RAG depende da capacidade de recuperar documentação relevante. No contexto de suporte técnico, a recuperação necessita ser precisa para fornecer soluções corretas o que é desafiador devido à complexidade e especificidade da documentação \parencite{ToroIsaza2024}.


No que toca recuperação por similaridades semanticas, \cite{soman2024observations} refere que o uso de embeddings para chunks de texto grandes (> 200 palavras) resulta em valores de similaridade artificialmente altos. Isso sugere que, mesmo quando as frases não são semanticamente parecidas, o modelo acha que são, apenas por serem longas. Contudo, documentação que usa grande número de abreviações e paragrafos para um tópico tornanam as obeservações mais relevantes. Além disso, concluiu-se palavras-chave mais próximas do começo de de uma frase são recuperadas com maior precisão.

Adicionalmente, estudos recentes revelam que sistemas RAG apresentam dificuldades significativas quando aplicados em ambientes empresariais. \textcite{RAGDoesNotWork2024} demonstram que, mesmo quando a resposta correta está presente no contexto, o sistema frequentemente falha em recuperá-la. Isso ocorre, em parte, devido ao desajuste entre a estrutura dos documentos técnicos (como FAQs, procedimentos, logs, etc.) e as estratégias tradicionais de segmentação em chunks.

Essa falha é confirmada por \textcite{SevenPoints2024}, que identificaram múltiplos pontos críticos no funcionamento de sistemas RAG, incluindo:

\begin{itemize}
    \item Falta de conteúdo: Quando a informação necessária não está no contexto, o sistema pode responder com conteúdos enganosos, sugerindo que sabe a resposta mesmo sem dados de apoio.
    \item Fraca classificação da documentação: Mesmo com a informação presente no contexto, ela pode não ser corretamente classificada e consequentemente não recuperada. 
 \item Informação não extraida: Caso haja informações contraditórias no contexto, o retriever pode apresentar falhas.
  \item Especificidade incorreta: O sistema pode gerar respostas muito vagas ou excessivamente específicas sem compreender especificamente o que foi solicitado.
\end{itemize}

Diversas técnicas de Finetunning podem ser utilizadas para contornar estas situações.

TODO melhorar daqui para baixo
Contextos maiores geram respostas mais precisas. A inclusão de metadados, como o nome do ficheiro e número do chunk, melhora a interpretação da informação recuperada. Modelos de embeddings open source também se mostram eficazes, especialmente em textos curtos. Para garantir resultados robustos, é essencial calibrar cuidadosamente o pipeline RAG — incluindo chunking, embeddings, recuperação e consolidação — além de manter uma monitorização contínua, dado que o sistema lida com entradas desconhecidas em tempo real.



\subsubsection{Integração com Sistemas Existentes}




https://arxiv.org/pdf/2409.13707

https://arxiv.org/pdf/2404.00657


\subsection{RQ2 - Como as frameworks ou bibliotecas disponíveis para desenvolvimento RAG se comparam em termos de arquitetura, flexibilidade e facilidade de utilização?}

Antes de apresentar cada tecnologia, importa referir que as informações aqui descritas foram obtidas com base na respetiva documentação oficial disponível nos websites de cada projeto. Nesta análise, cada framework será avaliada segundo três critérios: (i) arquitetura, (ii) flexibilidade e (iii) facilidade de utilização. Adicionalmente, serão também consideradas as formas de integração com bases de dados externas (vector stores ou document stores), bem como o suporte e comunidade em torno de cada projeto.


\subsubsection{LangChain}


https://www.ibm.com/think/topics/langchain

Fundada por Harrison Chase em 2022, LangChain é uma framework open-source de orquestração para o desenvolvimento de aplicações que fazem uso de LLMs. Está disponível em Python ou Javascript e oferece um ambiente centralizado para construir soluções que integram LLMs com diferentes fontes de dados externas e workflows de software.

É compatível como a maioria dos LLMs como GPT da OpenAI através da respetiva API Key e com modelos open-source como LLaMa e Google Flan-T5 através da plataforma Hugging Face. 
Oferece uma maneira comoda de manipular os Prompt Templates para gerar respostas consistentes e possibilita a organização e recuperação de dados contextuais externos. 

Referencia de chains: https://python.langchain.com/v0.1/docs/modules/chains/

Chains são a sua funcionalidade core. Estas representam sequências de chamadas, tanto para um LLM como para ferramentas adicionais, que permitem compor fluxos de processamento complexos de forma modular. LangChain Expression Language (LCEL) é uma linguagem declarativa que facilita a criação de Chains reutilizávies. LCEL fornece diversos construtores de Chains prontos para uso, como: 

\begin{itemize}
	\item \textit{create\_stuff\_documents\_chain}: Formata uma lista de documentos em um prompt para o LLM.
	\item \textit{create\_sql\_query\_chain}: Gera consultas SQL a partir linguagem natural.
	\item \textit{create\_history\_aware\_retrieve}: Utiliza o histórico de conversas para gerar consultas de busca mais precisas.
	\item \textit{create\_retrieval\_chain}: Integra recuperação de documentos relevantes com geração de respostas por LLM.
\end{itemize}


======================================================

TODO falar dos agents? 
TODO referir que existe integração com ferramentas externas tipo google search 
TODO referir o suporte com bases de dados vetoriais ou document store (formas de ligação entre informação externa e o sistema)
TODO suporte/comunidade

TODO ferramentas e ecossistema: 
TODO Falar Suporte a LangServe, LangSmith (monitoramento), LangGraph (para fluxos complexos).

TODO Comunidade mais ativa, integração com muitos serviços.


======================================================



\subsubsection{Haystack}


Haystack é uma framework open-source desenvolvida pela empresa alemã Deepset, com o objetivo de facilitar a construção de pipelines baseadas em LLMs, especialmente para serem usadas em casos de uso de pesquisa, como RAG, respostas a perguntas (question answering), classificação, extração de informação e pesquisa semântica em documentos, sendo a sua linguagem core o Python. A sua primeira versão surgiu em 2020, tendo recentemente evoluido para a sua versão 2.0, que levou à introdução de uma arquitetura completamente modular e orientada a componentes, com foco na flexibilidade, na reutilização e na fácil integração com serviços externos, como OpenAI, Hugging Face e outros.

A principal inovação do Haystack 2.0 reside na sua abordagem centrada em componentes. Cada componente representa uma unidade funcional independente, com uma responsabilidade bem definida dentro de uma pipeline. Esta arquitetura permite que os desenvolvedores construam soluções complexas de forma declarativa, combinando componentes reutilizáveis para formar pipelines personalizadas, robustas e facilmente escaláveis.

Por exemplo, os componentes de armazenamento de documentos, conhecidos como Document Stores, são responsáveis por armazenar e disponibilizar os documentos a serem consultados pelos restantes elementos da pipeline. Estes podem basear-se em tecnologias como Elasticsearch, Weaviate, Qdrant, entre outras, e suportam tanto índices tradicionais quanto vetoriais. Para alimentar estas bases de dados, o Haystack oferece Data Connectors e Indexers, que permitem extrair dados de múltiplas fontes (como diretórios locais, bases de dados ou APIs) e indexá-los com o formato e estrutura apropriados.

A recuperação da informação é realizada através dos Retrievers, que podem operar com base em métodos tradicionais (como BM25 TODO explicar isto na parte do RAG) ou com embeddings gerados por modelos de linguagem, permitindo uma recuperação densa e semântica. Após a recuperação, é possível utilizar Rankers para reordenar os documentos segundo critérios de relevância mais refinados. Em cenários em que se pretende gerar respostas completas com base na informação recolhida, os Generators entram em ação, utilizando modelos LLM para sintetizar respostas naturais e contextualizadas. Alternativamente, podem ser utilizados Prompt Nodes, que consistem em componentes configuráveis que enviam prompts para modelos de linguagem externos ou locais, oferecendo grande flexibilidade na forma como as instruções são formuladas.

Outro elemento importante da arquitetura são os Routers, que introduzem lógica condicional nas pipelines. Com eles, é possível definir fluxos alternativos com base nos resultados de componentes anteriores, facilitando a criação de pipelines mais inteligentes e adaptativas.

Todos estes componentes são integrados dentro de pipelines, que representam a sequência e lógica de execução entre os vários elementos da solução. As pipelines podem ser construidas de forma declarativa através de ficheiros YAML, ou programaticamente em Python, permitindo a configuração parâmetros, condições e integrações externas. 

O Haystack disponibiliza pipelines template que são adequadas para uso em contexto RAG, como a \textit{PredefinedPipeline.INDEXING} que importa ficheiros de texto, cria embeddings para cada um e armazena-os num \textit{InMemoryDocumentStore}. Adicionalmente temos a \textit{PredefinedPipeline.RAG} que usa os dados que foram previamente indexados para gerar respostas contextualizadas sobre essa informação. 



======================================================

https://medium.com/aimonks/haystack-an-alternative-to-langchain-carrying-llms-bf7c515c9a7e
https://haystack.deepset.ai/overview/intro
https://docs.haystack.deepset.ai/docs/pipeline-templates


TODO referir o suporte com bases de dados vetoriais ou document store (formas de ligação entre informação externa e o sistema)

TODO suporte/comunidade


Haystack:

Tem seu próprio servidor (Haystack REST API).

Ferramentas como Haystack UI para demonstrações.




\subsubsection{LlamaIndex}

TODO



\subsubsection{Spring AI}


O projeto Spring AI ficou disponível a público no inicio de 2024. Surgiu com o objetivo de possibilitar o desenvolvimento de soluções orientadas à inteligencia artificial sem grandes complicações. Foi inspirado em alguns dos projetos que foram referidos anteriormente, como LangChain e LlamaIndex e criado com a crença de não restringir o desenvolvimento deste tipo de soluções apenas para o ecossistema Python, mas também para ecossistemas JVM com o Spring Framework.

Através da sua arquitetura modular, o Spring AI suporta a criação de fluxos RAG personalizados ou a utilização de fluxos template, isto através da Advisor API. 

Dentro desta arquitetura, destaca-se o \emph{QuestionAnswerAdvisor}, cujo objetivo principal é obter informação diretamente de uma base de dados vetorial, construindo uma resposta exclusivamente com os dados recuperados. Esta abordagem ignora o conhecimento prévio embutido no LLM utilizado, assegurando que a resposta é fundamentada apenas nas fontes de dados específicos.

Este advisor recebe por parametro a base de dados vetorial (vectorStore) e opcionalmente um SearchRequest que permite a configuração do grau de similariade de pesquisa. No snippet \ref{lst:question-answer-adviser-0-8} é possível visualizar uma definição do grau de similaridade a 80\%.


\begin{lstlisting}[language=Java, caption={Configuração do grau de similaridade do QuestionAnswerAdvisor}, label={lst:question-answer-adviser-0-8}]
var qaAdvisor = new QuestionAnswerAdvisor(this.vectorStore,
        SearchRequest.builder().similarityThreshold(0.8d).build());
\end{lstlisting}


Além do \emph{QuestionAnswerAdvisor}, o Spring AI disponibiliza também o \emph{RetrievalAugmentationAdvisor}, que utiliza uma abordagem híbrida entre o conhecimento interno do LLM e a informação recuperada da base de dados vetorial. Neste caso, a resposta gerada pelo modelo pode incorporar tanto os dados recuperados quanto o conhecimento prévio do modelo base, permitindo um equilíbrio mais flexível entre a geração livre e a recuperação dirigida.

A utilização do \emph{RetrievalAugmentationAdvisor} segue uma configuração semelhante, permitindo a parametrização da consulta à base de dados vetorial, mas com a diferença fundamental de que o LLM poderá usar o contexto adicional para enriquecer a resposta final, ao invés de se limitar exclusivamente aos dados recuperados.

TODO exemplo do RetrievalAugmentationAdvisor

Complementarmente, o Spring AI organiza as operações de RAG em componentes denominados \emph{Modules}. Estes módulos são unidades configuráveis que encapsulam diferentes etapas de um fluxo de execução RAG, como a preparação de documentos (\emph{Document Reader Modules}), a transformação de texto em embeddings (\emph{Embedding Modules}), a pesquisa em bases vetoriais (\emph{Retriever Modules}) e a geração de respostas (\emph{LLM Modules}). Esta abordagem modular permite grande flexibilidade na definição de pipelines, podendo os desenvolvedores adaptar cada etapa do fluxo de acordo com os requisitos específicos da aplicação.


======================================================



TODO referir o suporte com bases de dados vetoriais ou document store (formas de ligação entre informação externa e o sistema)







\section{Related Work}

nao sei se fica bem aqui este topico



