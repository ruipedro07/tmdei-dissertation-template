% Chapter 1
% 
\chapter{Introduction} % Main chapter title
\label{chap:Chapter1} % For referencing the chapter elsewhere, use Chapter~\ref{Chapter1}


%-------------------------------------------------------------------------------
%---------
%
\section{Context} 


\section{Problem}

A Natixis é uma empresa do setor financeiro, parte do grupo bancário francês BPCE (Banque Populaire,
 Caisse d'Epargne). Ela atua principalmente em banca de investimentos, gestão de ativos, seguros e serviços
 financeiros especializados [1].

 Atualmente a equipa "B2C" da Natixis efetua tarefas diárias de suporte técnico no sistema pelo qual é
 responsável. Este é um sistema maioritariamente responsável pelo cálculo de risco de crédito bancário. O
 sistema respeita um fluxo bem definido sendo que diariamente correm diversos processos. Primeiramente
 vem a fase de alimentação, onde o sistema injeta dados provenientes de sistemas externos e popula as
 tabelas brutas da base de dados. De seguida vem o processo de enriquecimento dos dados, onde os
 mesmos são analisados em termos de qualidade, sendo aplicadas regras de negócio para os alterar e
 armazenar. Todo esse processo é auditado em ficheiros e/ou tabelas específicas na base de dados. Por fim
 vem o processo de cálculo, que pode ser de vários tipos, como por exemplo RC B3 (Cálculo do Capital
 Ponderado pelo Risco - B3) que nos sistemas bancários se refere, normalmente, aos requisitos de capital
 regulamentares definidos pelo acordo de Basel III [2].
 Todo este processo é controlado diariamente pelo Control-M. O Control-M é uma ferramenta de automação
 de workload e gestão de jobs [3]. Ele é amplamente utilizado para agendar, monitorizar e gerir processos
 batch. A equipa de suporte é responsável por gerir esta chain, sendo que cada processo (alimentação,
 enriquecimento e cálculo) é composto por diversos batchs que executam maioritariamente código Perl e
 Java.

 Falhas na chain são comuns de acontecer e podem ter diversos motivos, como erros nos dados externos que
 violem as regras de negócio, problemas de código provenientes de desenvolvimentos recentes, entre outros.
 A equipa é também responsável por lançar processos "on demand", facilitar o esclarecimento de questões
 relacionadas com regras de negócio aos utilizadores/partes interessadas, entre outras atividades.
 Devido à dimensão do software e à quantidade de diferentes processos envolvidos, por vezes a atividade de
 suporte torna-se uma tarefa bastante complicada para a equipa responsável.



\section{Objectives}

Explorar Retrieval-augmented generation (RAG) juntamente com Large Language Model (LLM) para construir
 um sistema que forneça auxílio à tomada de decisão nas atividades de suporte, pela geração de informação
 contextualizada.
 Retrieval-Augmented Generation (RAG) combina a geração de texto por Large Language Models (LLMs) com
 recuperação de conhecimento externo, permitindo respostas mais informadas e específicas [4].
 Os LLMs, como GPT, Llama e Mistral, são modelos que possuem um vasto conhecimento, mas não
 conseguem aceder a informações de domínios específicos. O RAG resolve essa limitação ao combinar uma
 base de dados vetorial para armazenar representações semânticas de documentos e consultas, permitindo
 que o modelo recupere a informação mais relevante antes de gerar uma resposta. Essas bases de dados
 vetoriais utilizam embeddings, que representam o significado semântico do texto em um espaço
 multidimensional, possibilitando pesquisas mais eficientes e contextuais [4,5].
 A aplicação do RAG neste contexto permitirá otimizar a eficiência da equipa de suporte, reduzindo o tempo
 gasto na procura de informação e facilitando a resolução de incidentes.

 Funcionalidades a explorar:

 - Facilitar e agilizar a obtenção de informação relevante para resolução de determinado problema. O
 software deverá fornecer insights de passos a tomar com base no conhecimento que possui.
 - Automatizar a extração de conhecimento de fontes diversas, como manuais e logs de execução,
 integrando-se com as ferramentas utilizadas pela equipa (p.e. Control-M, Confluence, Outlook e Teams).
 - Auxiliar na resolução de problemas ao sugerir soluções com base em experiências anteriores e na
 análise de padrões de erro.

 O projeto irá obedecer à seguinte ordem cronológica:

 - Primeira fase: Análise de KPIs históricos na vertente de resolução de incidentes.
 - Segunda fase: Pesquisa / análise de trabalhos relacionados e boas práticas.
 - Terceira fase: Levantamento de requisitos.
 - Quarta fase: Implementação da solução.
 - Quinta fase: Testes.

\section{Methodology}

\section{Work Plan}

\section{Document Structure}

